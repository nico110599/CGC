%PREAMBOLO
\documentclass[a4paper, 12pt]{report}
\usepackage[italian]{babel}
\usepackage{graphicx}
\usepackage{amsmath,amssymb}
\usepackage{amsbsy}
\usepackage{xcolor}
\usepackage{enumitem}
\usepackage{multicol}
\renewcommand{\footnoterule}{
  \kern -3pt
  \hrule width \textwidth height 1pt
  \kern 2pt
}%ALLUNGA LINEA PIE DI PAGINA
\setcounter{tocdepth}{3}%AGGIUNGE SUBSUBSECTION ALL'INDICE
%INIZIO
\begin{document}
\title{
\textbf{Cloud Green Computing}}
\author{Stefano Piccoli}
\date{\today}
\maketitle
\tableofcontents
    \chapter{IaaS (Infrastructure as a Service)}
      \section{Virtualizzazione}
        \paragraph{}La \textbf{virtualizzazione} rende possibile al sistema operativo di un server di eseguire su uno \textbf{strato virtuale} (\textbf{Hypervisor}).\\
        Questo permette di eseguire molteplici \textbf{macchine virtuali}, ognuna con il proprio sistema operativo, sullo stesso server fisico.
      \section{Hypervisor}
        \paragraph{}L'\textbf{hypervisor} crea lo strato di \textbf{virtualizzazione} che rende la virtualizzazione server possibile e contiene la \textbf{Virtual Machine Manager (VMM)}.
        \paragraph{Tipologie}
        \begin{itemize}
          \item \textbf{Type 1}: caricata direttamente sull'hardware, può eseguire più virtual server, usato per data center o server
            \subitem $\circ$ Hyper-v 
            \subitem $\circ$ ESX/ESXi 
            \subitem $\circ$ XenServer
          \item \textbf{Type 2}: caricata in un sistema operativo eseguito sull'hardware, greater overhead, usato per desktop e laptop
            \subitem $\circ$ Workstation 
            \subitem $\circ$ Virtual Server
            \subitem $\circ$ Fusion
        \end{itemize}
      \section{Amazon Elastic Compute Cloud 2 (EC2)}
      \begin{itemize}
        \item Mette a disposizione server virtuali (\textbf{istanze}) in modo semplice, veloce ed economico 
        \item Scelta tipo istanza e template da utilizzare (Windows/Linux) e numero istanze
        con AWS management console (o librerie SDK)
        \item \textbf{Opzioni di pagamento}: on demand, istanze riservate, istanze spot
        \item \textbf{Sicurezza} (Virtual Private Cloud - VPC
        \item \textbf{Storage persistente}: Amazon Elastic Block Store (EBS)
        \item \textbf{Autoscaling}
      \end{itemize}
      \section{Amazon Simple Storage Service (S3)}
      \begin{itemize}
        \item Fornisce uno \textbf{storage sicuro e facile} da usare
        \item Diverse \textbf{classi di memorizzazione} (standard / standard infrequent access / glacier)
        \item \textbf{Controllo} configurabile di \textbf{accesso ai dati}
      \end{itemize}
      \section{Amazon Elastic Block Store (EBS)}
      \begin{itemize}
        \item Blocco persistente di archiviazione di volumi di memoria usato con le istanze di Amazon EC2
        \item Ogni volume di Amazon EBS viene automaticamente replicato senza la sua Aviabilty Zone in modo da offrire alta disponibilità e durata. 
      \end{itemize}
      \section{Dropbox exodus}
        \begin{itemize}
          \item I primi 8 anni della sua vita archiviava miliardi di file su Amazon S3
          \item Tra il 2014 e 2016 ha costruito la propria rete di server ideata dai propri ingegneri per spostare i dati
        \end{itemize}
        \subsection{L'esodo}
        \begin{itemize}
          \item Hardware propietario che archivierà petabyte di dati
          \item Nuovo codice ("Magic Pocket")
          \item Installare 50 rack di hardware al giorno
          \item Completare lo spostamento prima della scadenza del contratto con Amazon per evitare un rinnovo
        \end{itemize}
        \subsection{Conclusioni}
        \paragraph{}Dropbox è riuscita a completare lo spostamento con successo entro i tempi previsiti.
    \chapter{Container}
    \paragraph{}I \textbf{containers} sono un meccanismo di virtualizzazione differente dalle Virtual Machines poichè 
    permetto di avere più istanze \textbf{isolate} e \textbf{volatili} che scompaiono quando interrotte.\\
    I containers sono \textbf{leggeri}, \textbf{veloci}, più \textbf{semplici da buildare} ma \textbf{meno sicuri} delle Virtual Machines.
    \section{Docker}
    \paragraph{}\textbf{Docker} è un'azienda che ha realizzato una piattaforma che permette di \textbf{eseguire una applicazione in ambiente "isolato"}.\\
    Docker sfrutta la \textbf{virtualizzazione basata sui container} per eseguire in maniera isolata diverse \textbf{GUEST INSTANCES} sullo stesso sistema operativo.
    \subsection{Caratteristiche}
    \begin{itemize}
      \item \textbf{Portabilità}: il software può essere impacchettato in \textbf{images}, file read only che può essere mandato in esecuzione da docker e creare quindi il container
      \item Possono avere più istanze separate degli spazi utente (\textbf{containers})
      \item \textbf{Interfaccia} utente \textbf{semplificata}
      \item \textbf{Svantaggio}: sono meno isolati delle macchine virtuali, \textbf{condividono le risorse di sistema}
    \end{itemize}
    \subsection{Componenti}
    \begin{itemize}
      \item \textbf{Docker Engine}: permette di creare e mandare in esecuziuone container
      \item \textbf{Docker Hub}: repository enorme che contiene molte immagini di container
      \item \textbf{Docker Swarm Mode}: permette di eseguire un container su più docker host e divide gli swarm node in manager e worker, permettendo una \textbf{gestione dichiarativa} della nostra \textbf{applicazione}
      \item \textbf{Images}: \textbf{template di sola lettura} usati per creare container, registrate in registry
      \subitem $\circ$ \textbf{Stratificazione}: ogni strato può essere a sua volta una immagine 
      \item \textbf{Registry}: \textbf{strutture di repository} che contengono insiemi di immagini per diverse versioni del sw
    \end{itemize}
    \subsection{Comandi}
    \begin{itemize}
      \item \textbf{PULL}: tiro un'immagine dal registry alla macchina
      \item \textbf{RUN}: viene creato il container dell'immagine
      \item \textbf{COMMIT}: salvare una nuova immagine
      \item \textbf{PUSH}: caricare una immagine nel registry
      \item \textbf{BUILD}: si crea un dockerfile che permette di creare un'immagine automaticamente
    \end{itemize}
    \subsection{Swarm mode}
    \begin{itemize}
      \item I nodi possono agire da \textbf{managers}, delegando tasks, o \textbf{workers}, eseguendo task assegnati.
      \item È possibile definire lo \textbf{stato dei vari servizi} nello stack dell'applicazione, incluso il numero di \textbf{task da eseguire in ogni servizio}
      \item \textbf{Swarm manager}:
        \subitem $\circ$ assegna ad ogni servizio nello swarm un unico DNS name
        \subitem $\circ$ bilancia il carico dei container in esecuzione 
        \subitem $\circ$ monitora lo stato del cluster e lo allinea con quello desiderato
    \end{itemize}
    \chapter{PaaS (Platform as a Service)}
      Servizio che fornisce hardware e software per lo sviluppo di applicazioni. L'utente deve fornire solo l'applicazione e i dati
    \paragraph{Vantaggi}
    \begin{itemize}
      \item Facilità di gestione e modifica dell'applicazione
      \item Facilità nell'adottare nuove tecnologie
    \end{itemize}
    \paragraph{Rischi}
    \begin{itemize}
      \item Disponibilità del servizio: l'interruzione del servizio da parte del fornitore comporta un immediato disservizio
      \item Vendor lock-in: difficoltà di cambiare servizio da parte del cliente
    \end{itemize}
      \section{Heroku}
      \paragraph{}\textbf{Heroku} è una piattaforma cloud basata su \textbf{container} con servizi integrati e un potente ecosistema che permette il deployment e running di applicazioni.
      \subsection{Dynos}
      \paragraph{}I \textbf{dynos} sono container Linux virtualizzati, Heroku trasforma l'applicazione utente in diversi \textbf{dynos}.
      \paragraph{Vantaggi}
      \begin{itemize}
        \item Scalabilità 
        \item Evitare di gestire l'infrastruttura
      \end{itemize}
      \paragraph{Premium:}
      \begin{itemize}
        \item \textbf{Scaling}
        \item \textbf{Autoscaling}: permette di inserire politiche per quando usare lo scaling
      \end{itemize}
      \subsection{Buildtime}
      \paragraph{}Per sviluppare una applicazione Heroku richiede:
      \begin{itemize}
        \item \textbf{Codice sorgente}
        \item \textbf{Lista di dipendenze}
        \item \textbf{Procfile}: file di testo che indica quale comando usare per far eseguire l'applicazione
      \end{itemize}
      Slug: Un insieme di codice sorgente, dipendenze, supporto per output, etc...
      Stack: Sistema operativo Ubuntu
      \subsection{Runtime}
      Nel \textbf{runtime} si prende lo slug e lo stack e vengono creati i dynos, che rappresentano le istanze utente, il dyno manager fa partire i container con il comando specificato dall'utente.
      \subsection{Esempio}
      \begin{enumerate}
        \item Applicazione riceve richiesta
      \end{enumerate}
      %SOTTO GRAFICO
      \subsection{Add-ons}
      \paragraph{}Gli \textbf{add-ons} sono funzionalità fornite da Heroku che possono essere aggiunte facilmente all'applicazione.
    \section{Altri PaaS}
    \begin{itemize}
      \item Microsoft Azure
      \item OpenShift
    \end{itemize}
  \chapter{Modelli di Business}
    \paragraph{}Un \textbf{business model} descrive il razionale di come una azienda \textbf{crea}, \textbf{consegna} e \textbf{acquisisce valore}.
    \begin{itemize}
      \item \textbf{Customer Segments}: il gruppo di persone o organizzazioni a cui il servizio mira di raggiungere
      \item \textbf{Valuer Propositions}: cosa rende speciale il servizio
      \item \textbf{Channels}: le modalità in cui la compagnia raggiunge il cliente
      \item \textbf{Customer Relationships}: tipo di relazione che la compagnia stabilisce col cliente 
      \item \textbf{Revenue Streams}: il flusso di entrate che la compagnia genera da ogni segmento di clientela
      \item \textbf{Key Resources}: le risorse più importanti richieste per il modello di business 
      \item \textbf{Key Activities}: le attività più importanti che la compagnia deve svolgere
      \item \textbf{Key Partners}: la rete di fornitori e partners per il business 
      \item \textbf{Cost Structure}: i costi che si incontrano per operare nel modello di business
    \end{itemize}
      \section{Business innovation}
        \begin{itemize}
          \item \textbf{Resource-driven}: ha origine da \textbf{infrastrutture o partner gia esistenti} usate per espandere o trasformare il business model 
          \item \textbf{Offer-driven}: crea nuova value proposition che influenza altri ambiti del business model
          \item \textbf{Custmer-driven}: basato sulle necessità del cliente, accesso facilitato o aumento di convenienza
          \item \textbf{Finance-driven}: guidata dal \textbf{revenue stream}, meccanismo di prezzi o riduzione dei cost structure
        \end{itemize}
\end{document}