%PREAMBOLO
\documentclass[a4paper, 12pt]{report}
\usepackage[italian]{babel}
\usepackage{graphicx}
\usepackage{amsmath,amssymb}
\usepackage{amsbsy}
\usepackage{xcolor}
\usepackage{enumitem}
\usepackage{multicol}
\renewcommand{\footnoterule}{
  \kern -3pt
  \hrule width \textwidth height 1pt
  \kern 2pt
}%ALLUNGA LINEA PIE DI PAGINA
\setcounter{tocdepth}{3}%AGGIUNGE SUBSUBSECTION ALL'INDICE
%INIZIO
\begin{document}
\title{
\textbf{Cloud Green Computing}}
\author{Stefano Piccoli}
\date{\today}
\maketitle
\tableofcontents
    \chapter*{Introduzione}
    \chapter{PaaS (Platform as a Service)}
      Servizio che fornisce hardware e software per lo sviluppo di applicazioni. L'utente deve fornire solo l'applicazione e i dati
    \paragraph{Vantaggi}
    \begin{itemize}
      \item Facilità di gestione e modifica dell'applicazione
      \item Facilità nell'adottare nuove tecnologie
    \end{itemize}
    \paragraph{Rischi}
    \begin{itemize}
      \item Disponibilità del servizio: l'interruzione del servizio da parte del fornitore comporta un immediato disservizio
      \item Vendor lock-in: difficoltà di cambiare servizio da parte del cliente
    \end{itemize}
      \section{Heroku}
      \paragraph{}\textbf{Heroku} è una piattaforma cloud basata su \textbf{container} con servizi integrati e un potente ecosistema che permette il deployment e running di applicazioni.
      \subsection{Dynos}
      \paragraph{}I \textbf{dynos} sono container Linux virtualizzati, Heroku trasforma l'applicazione utente in diversi \textbf{dynos}.
      \paragraph{Vantaggi}
      \begin{itemize}
        \item Scalabilità 
        \item Evitare di gestire l'infrastruttura
      \end{itemize}
      \paragraph{Premium:}
      \begin{itemize}
        \item \textbf{Scaling}
        \item \textbf{Autoscaling}: permette di inserire politiche per quando usare lo scaling
      \end{itemize}
      \subsection{Buildtime}
      \paragraph{}Per sviluppare una applicazione Heroku richiede:
      \begin{itemize}
        \item \textbf{Codice sorgente}
        \item \textbf{Lista di dipendenze}
        \item \textbf{Procfile}: file di testo che indica quale comando usare per far eseguire l'applicazione
      \end{itemize}
      Slug: Un insieme di codice sorgente, dipendenze, supporto per output, etc...
      Stack: Sistema operativo Ubuntu
      \subsection{Runtime}
      Nel \textbf{runtime} si prende lo slug e lo stack e vengono creati i dynos, che rappresentano le istanze utente, il dyno manager fa partire i container con il comando specificato dall'utente.
      \subsection{Esempio}
      \begin{enumerate}
        \item Applicazione riceve richiesta
      \end{enumerate}
      %SOTTO GRAFICO
      \subsection{Add-ons}
      \paragraph{}Gli \textbf{add-ons} sono funzionalità fornite da Heroku che possono essere aggiunte facilmente all'applicazione.
    \section{Altri PaaS}
    \begin{itemize}
      \item Microsoft Azure
      \item OpenShift
    \end{itemize}
  \chapter{Modelli di Business}
    \paragraph{}Un \textbf{business model} descrive il razionale di come una azienda \textbf{crea}, \textbf{consegna} e \textbf{acquisisce valore}.
    \begin{itemize}
      \item Segmenti di mercato: definisce il gruppo di persone o organizzazioni a cui il servizio mira di raggiungere
      \item Proposta:
      \item Canali: Le modalità in cui la compagnia raggiunge il cliente
      \item Relazioni con i clienti
      \item Flussi di guadagno:
      \item Risorse chiave
      \item Attività chiave
      \item Partner chiave
      \item Struttura di costo
    \end{itemize}
      \section{Customer insights}
    Fulfillment:
    
\end{document}